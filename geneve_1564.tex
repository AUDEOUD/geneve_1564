\documentclass[twocolumn,paper=8in:14in,pagesize=pdftex,12pt]{scrbook}
\areaset[0.5in]{5in}{12in}

\usepackage{bigfoot}
\usepackage{ragged2e}

\usepackage{fontspec}
\usepackage{xunicode}
\defaultfontfeatures{Mapping=tex-text}
%\setmainfont{Linux Libertine O}
\setmainfont[RawFeature={+ss02,+cv01,+ss05,+dlig},ItalicFeatures={RawFeature=+cv04}]{EB Garamond}
\newfontfamily\booktitlefont[LetterSpace=40,WordSpace=7,RawFeature={+ss02,+cv01,+dlig}]{EB Garamond}
\newfontfamily\spacedfont[LetterSpace=20,WordSpace=7,RawFeature={+ss02,+cv01,+dlig}]{EB Garamond}

\usepackage{graphicx}
\usepackage{lettrine}


% Verse references
% To be adjusted
\usepackage{bibleref-french}
\renewcommand*{\BRchvsep}{.}%
\renewcommand*{\BRvsep}{,}%
\setbooktitle{Job}{Iob}%
\setbooktitle{Ps}{Pseau.}%
\setbooktitle{Jn}{Iean}%
\setbooktitle{He}{Hebr.}%
\renewcommand{\BRbooktitlestyle}{\textit}


% Environments
\usepackage{setspace}
\newenvironment{comment}
  {\begin{spacing}{0.8}\itshape\scriptsize\setlength\hspace{-1em}}
  {\end{spacing}}



% Footnotes
\DeclareNewFootnote{default}[alph]
\DeclareNewFootnote{chapter}[Roman]
\DeclareNewFootnote{verse}[arabic]


\setlength{\marginparwidth}{4.3em}% adjust to your document's needs

\newcommand{\chapnote}[2][0pt]{%
   \footnotemarkchapter%
   \marginpar{\vspace{#1}\tiny\textsuperscript{\thefootnotechapter}\justifying#2}}

\newcommand{\sidenote}[2][0pt]{%
   \footnotemark%
   \marginpar{\vspace{#1}\tiny\textsuperscript{\thefootnote}\justifying#2}}

\newcommand{\versenote}[2][0pt]{%
   \footnotemarkverse%
   \marginpar{\vspace{#1}\tiny\textsuperscript{\thefootnoteverse}\justifying#2.}}

\newcommand{\fakenote}[3][\space]{%
   \par\noindent#1\textsuperscript#2\justifying#3
}


% Bible verses
\newcounter{verse}
\newcommand{\bverse}{%
  \addtocounter{verse}{1}
  \theverse\quad
}

\newcommand{\bversenopar}[1][\indent]{%
   \addtocounter{verse}{1}\\#1\theverse~
}

\newcommand{\bversenonum}{%
   \addtocounter{verse}{1}
   \par
}


\usepackage{titlesec}
% Use fourier ornaments
\usepackage{fourier-orns}

% Bible books
\newcommand{\bbook}[3][]{%
  \chapter[#1]{#2,\\\Large #3\\\aldine}
}
\titleformat{\chapter}[hang]%
   {\centering\huge}%
   {}%
   {0pt}%
   {}

% Bible chapters
\newcommand{\bchapter}{%
   \setcounter{verse}{0}%
   \section{}{}
}

\renewcommand{\thesection}{\roman{section}}
\titleformat{\section}[hang]%
   {\booktitlefont\centering}%
   {\textsc{chapitre\ \thesection}.}%
   {5pt}%
   {}
\titlespacing*{\section}
  {0pt}{0pt}{0pt}


\setlength{\parindent}{-5pt}
\setlength{\parskip}{4pt}


\linespread{0.9}
\setlength{\columnsep}{6mm}

\begin{document}

\pagestyle{empty}

\twocolumn[
\begin{@twocolumnfalse}
\bbook{Le premier livre de Moyse}{Dict Genese.}

\begin{center}
\booktitlefont\textsc{argument.}
\end{center}
\begin{center}
\parbox{4.65in}{
\begin{comment}
Ce premier livre comprendre l'origine \& causes de toutes choses, principalement
 la creation de l'homme, qu'il a esté \linebreak
 du commencement, sa cheute \& relevement~: comment d'un tous ont esté
 procreés, \& pour leurs enormes pechés Dieu \linebreak
 les a consumés, par le deluge, reservé huict, dont la semence a rempli toute
 la terre. Puis il descrit les vies, faicts, reli- \linebreak
 gion, \& lignees des saints Patriarches, qui ont vescu devant la Loy~:
 Les benedictions, promesses, \& alliances du Sei- \linebreak
 gneur faictes avec iceux~: Comment de le la terre de Chanaan
 sont descendus en Egypte. Aucuns ont appelé ce livre, le \linebreak
 livre des Iustes.  Toutefois ceci a obtenu entre nos predecesseurs \& nous,
 qu'il est appelé Genese, qui est un mot Grec, \linebreak
 signifiant generation \& origine~: d'autant
 qu'en icelui est descrite l'origine \& procreation de toutes choses~:
 \& nommément des Peres anciens, qui ont esté tant devant qu'apres le deluge,
 \& eu esgard à {\spacedfont\emph{\textsc{iesus christ}}} descen- \linebreak
 du d'iceux selon la chair.
\end{comment}
}
\end{center}
\vspace{1cm}
\end{@twocolumnfalse}
]


\bchapter

\begin{comment}
 \footnotemarkchapter{}Creation du ciel \& de la terre,
 II, 10. \& de tout ce qui y est com\-prins.
 3.14. De la lumiere aussi, 26 \& de l'homme,
 18 {\addfontfeature{RawFeature=+swsh}Auquel} tout est assuietti.
 2.2. 18 Dieu benit toutes ses \oe{}uvres,
 31 qu'il a accomplies en six iours.
\end{comment}

\vspace{\baselineskip}

% TBD: span lettrine on 10 lines
\bversenonum \lettrine[lines=10,image=true]{D}{}%
 \footnotemarkverse{}Ieu
 \footnotemark{}crea 
 \footnotemark{}au com \linebreak
 mence - ment
 \footnotemark{}le ciel \& la terre.
\bversenopar[]Or la\linebreak
 terre eſ-\linebreak toit sans forme, \& \linebreak
 vuide, \& les tenebres estoyent sur les
abysmes~: \& l'Esprit de Dieu
 \footnotemark{}estoit
 espandu par dessus les eaux.



\bverse Adonc Dieu dît,
 \footnotemarkverse{}Qu'il y ait lumie\-re.
 \footnotemark{}Et la lumiere fut.

\bverse Et Dieu vid \~q la lumiere estoit bon\-ne~: \& separa la lumiere des tenebres.

\bverse Et Dieu appela la lumiere iour,\& les
 tenebres nuict. Lors fut faict le
 \footnotemark{}soir \& le matin 
 du premier iour.

\bverse ¶ Puis Dieu dît,
 \footnotemarkverse{}Qu'il y ait une
 \footnotemark{}eſ\-tendue entre les eaux,
 \& qu'elle separe les
 \footnotemark{}eaux d'avec les eaux.

\bverse Dieu donc fit l'estendue, \& divisa\linebreak

% Notes for the page
\marginpar{\vspace{-5.7in}\tiny
   \fakenote{I}{Ce premier cha\-pitre est fort difficile~: \& pour cette cause,
 il estoit defendu entre les Hebrieux de le lire \& interpreter
 devant l'aage de trente ans.}
   \fakenote{a}{Fit de rien, \& sans aucune matiere.}
   \fakenote{1}{\bibleverse{Job}(38:4), \bibleverse{Ps}(33:6), \& 89.13.}
   \fakenote{b}{Tout premierement, \& av\~at qu'il y eut aucune
 creature, \bibleverse{Jn}(1:10).}
   \fakenote{2}{\bibleverse{He}(11:3).}
   \fakenote{c}{Le ciel \& la terre,
 les eaux, les abysmes, se prennent ici pour vne mesme chose~:
 asç. pour une matiere c\~ofuse \& sans forme, \~q Dieu forma
 \& agença apres par sa Parole.}
   \fakenote{d}{Ou, se mouvoit. C'est, soustenoit et conservoit
 en son estre cette matiere confuse.
 Car il est impossible, \~q aucune chose apres avoir esté
 faictes, puisse subsister un seul moment, si Dieu ne la soustient
 \& c\~oserve par sa vertu, \bibleverse{Ps}(130:).}
   \fakenote{e}{Cette lumiere n'estoit point encore au soleil, car
 il n'avoit pas esté creé, mais estoit en la main de Dieu,
 ay\~at son ordre successif avec les tenebres,
 pour faire le iour \& la nuict \& ce iusques au
 quatrieme iour, que Dieu fit le soleil pour estre
 ministre \& dispensateur de cette lumiere, avec
 la lune \& estoilles.}
   \fakenote{3}{\bibleverse{Ps}(33:6) \& 136.5.}
   \fakenote{f}{Ici est la cause}
}
 \pagebreak

 \noindent les eaux, qui estoyent sous
 l'estendue, d'avec celles, qui estoyent sur l'esten\-due. Et fut ainsi faict.

\bverse Et Dieu appela l'estendue, Ciel.
 Lors fut faict le soir \& le matin du second iour.

\bverse ¶ Puis Dieu dît, 
 \footnotemarkverse{}~\footnotemark{}Que les eaux,
 qui sont sous le ciel,
 soyent assemblees en un lieu, \& que le sec apparoisse. Et fut ainsi faict.

\bverse Et Dieu appe\char"A749 ale sec,Terre,\& l'assem \linebreak
 blee des eaux, mers.
 Et Dieu vid que celà estoit bon.

\bverse Et Dieu dît, Que la terre produise verdure, herbe produisant semence,
 \& arbre fruictier, faisant fruict selon son espece, lequel ait sa sem\~ece
 en soy-meſ\-me sur la terre. Et fut ainsi faict.

\bverse La terre d\~oc produisit verdure, her\-be produisant sem\~ece
 selon son espece, \& arbre sans fruict, lequel avoit sa \linebreak
  semence en soymesme selon son espe- \linebreak
 ce. Et Dieu vid que celà estoit bon.

\bverse Lors fut faict le soir \& le matin du troisieme iour.

%skip j counter
% TBD: do it better
\addtocounter{footnote}{1}

\bverse ¶ Apres Dieu dît,\footnotemarkverse{}\,\footnotemark{}Qu'il y ait lumi \linebreak
 naires en l'estendue du ciel, pour
 sepa\-rer la nuict du iour~: \& soy\~et en
 \footnotemark{}signes,

\raggedleft a\quad en

% Notes for the page
\marginpar{\vspace{-5.7in}\tiny
 % Continue note f from left column
 pourquoy les Hebrieux c\~omencent
 le iour naturel le soir apres le soleil couchant.
   \fakenote{g}{Ce mot d'Est\~e \linebreak
 due, compr\~ed tout ce qui
 se voit par dessus nous, t\~at en la region celeste,
 qu'elementaire.}

   \fakenote{4}{\bibleverse{Ps}(33:7).}
   \fakenote{h}{Il est ici parlé de deux manieres
 d'eaux~: asçavoir, celles q sont sous
 l'estendue, comme la mer, les fleuves,
 \& autres qui sont sur la terre \& cel- \linebreak
 les, qui sont sur l'estendue,
 comme sont les nuees pleines d'eau
 ça haut en l'air par dessus nous.
 Dieu a mis entre ces deux forces d'eaux
 une gr\~ade estendue, qu'on appelle le ciel~:
 de là nous appelons les oiseaux du ciel.}
   \fakenote{i}{Ceci apparti\~et au sec\~od iour,
 auquel Dieu separa, \& fit apparoir la terre du milieu des eaux.}
   \fakenote{k}{Il institue un nouvel ordre
 en nature, quand il faut \& ordonne le soleil distributeur
 de cette lumiere qu'il avoit creée avant lui, \& avant la lune \& les eſ- \linebreak
 toilles.}
   \fakenote{5}{\bibleverse{Ps}(136:7)}
   \fakenote{l}{C'est pour si- \linebreak
 gnifier diverses di- \linebreak
 spositions que les corps \~iferieurs se- \linebreak
 lon l'ordre de na- \linebreak 
 ture ont des corps \linebreak
 celestes, c\~ome cau \linebreak
 ses sec\~odes ordon \linebreak
 nees de Dieu à ce- \linebreak 
 là.  En quoy tou- \linebreak
 tesfois faut fuir cu- \linebreak
 riosité \& supersti- \linebreak
 tion \~q les h\~omes \linebreak
 ont c\~otrouvee sur \linebreak
 celà.}
}

\end{document}

